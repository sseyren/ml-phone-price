\documentclass[a4paper,12pt]{article}

\usepackage[margin=1in]{geometry}
\usepackage{ucs}
\usepackage[utf8]{inputenc}
\usepackage{amsmath}
\usepackage[turkish]{babel}
\usepackage{fontenc}
\usepackage{hyperref}
\usepackage{graphicx}
\usepackage{tikz}

\title{\texttt{BMM4216} Örüntü Tanıma \\ Ara Sınav Ödevi}
\author{\texttt{201713709039} -- Serhat Seyren}
\date{4 Mayıs 2021}

\begin{document}

\maketitle

\begin{abstract}
Bu ödev, tamamen web tarayıcısında çalışabilen bir makine öğrenmesi uygulamasıdır.
Uygulamanın ana amacı, akıllı telefonları özelliklerine göre fiyatlarını tahmin etmek, uygun fiyat kategorisi ile eşleştirmektir.
Kullanıcı bir web sayfasını ziyaret ederek uygulamaya erişebilir, akıllı telefonun özelliklerini girerek, o cihaz için öngörülen fiyat kategorisini elde edebilir.
Web tarayıcılarının içerisinde çalışabilmesi için tüm süreç JavaScript programlama dili ile yapılmıştır.

Uygulama: \url{https://thesseyren.github.io/ml-phone-price/}

Kaynak Kod(GitHub): \url{https://github.com/thesseyren/ml-phone-price}
\end{abstract}

\tableofcontents

\newpage


\section{JavaScript Programlama Dili ve Ortamı}

Bu proje için yapay zeka uygulamaları için diğer alternatiflerine oranla (Python, Weka, MATLAB) daha az bilinen/kullanılan bir programlama dili seçtim.
Bunun temel sebebi, bu sayede uygulamanın doğrudan kullanıcının tarayıcısında çalıştırlabilecek olması.
Son kullanıcı sadece bir URL adresini ziyaret ederek uygulamayı kullanabilecek. Bu kısımda JavaScript'i ve ortamını tanıtmak istiyorum.

\subsection{Web Tarayıcıları ve Node.js}
JavaScript programlama dilinin en temel amacı statik web sayfalarına dinamiklik kazandırmak.
Örneğin bugün Facebook'u ziyaret ettiğinizde, size biri mesaj gönderdiğinde ekranınıza bildirim düşüyorsa; bunun mümkün olabilmesinin arkasında JavaScript betikleri yatıyor.
İlk olarak 1995 yılında Brendan Eich tarafından geliştirilen bu betik dili, dönemin web tarayıcısı olan Netscape'den beri tarayıcıların içinde yer alıyor.

JavaScript dili, web tarayıcısının bir parçası olduğundan, diğer programlama dillerinin yapabildiği pek çok şeyi tek başına yapamaz.
Örneğin kullanıcının dosya sistemine erişebilmesi, yeni bir dosya ekleyip silebilmesi gibi şeyler mümkün değil.
İlk defa 2009 yılında JavaScript'i kendi başına bir programlama dili olarak kullanılabilmesini sağlayan bir proje duyuruldu: Node.js.
Node.js sayesinde JavaScript dili, gerekli tüm standart kütüphaneler eklenerek, kendi başına kullanılabilecek bir programlama diline dönüştü.
Yıllarca web tarayıcılarına çalışan JavaScript, Node.js sayesinde web sunucuları üstünde, son kullanıcının bilgisayarında çalışmaya başladı.

\subsection{Paketler/Kütüphaneler}

Tabi, günümüz programlama dillerinin her birinde en az bir paket yöneticisi, kütüphane kurucusu mevcut.
Node.js için de aynı durum geçerli.
Node.js için bu ikisi ``npm'' ve ``yarn''.
Tabii bu ikisine alternatifler de mevcut ancak en popülerleri bunlar.

Bu projede kullandığım kütüphaneleri şöyle özetleyebilirim:

\begin{description}
 \item[\href{https://c2fo.github.io/fast-csv/}{fast-csv}] CSV dosyalarını okuyabilmemizi sağlıyor.
 \item[\href{https://mljs.github.io/matrix/}{ml-matrix}] Matris kütüphanesi. Veriyi bir matrise atmak ve üzerinde işlemler yapabilmemizi sağlıyor.
 \item[\href{https://github.com/mljs/knn}{ml-knn}] Genel amaçlı KNN\footnote{K-Nearest Neighbors} algoritması.
 \item[\href{https://github.com/mljs/confusion-matrix}{ml-confusion-matrix}] Eğitilmiş model için karışıklık matrisini\footnote{Confusion Matrix} ve F1 skorlarını sağlıyor.
 \item[\href{https://reactjs.org/}{react}] Tüm bu çalışmayı SPA\footnote{Single Page Application} getirecek frontend framework'ü.
 \item[\href{https://material-ui.com/}{material-ui}] Uygulamanın arayüzünde bulunan buton, düğme, anahtar, kart gibi görsel parçaların bulunduğu paket.
\end{description}

Kullanabileceğim kütüphanelere bakınırken, Python ortamından tanıdığımız popüler bir kütüphane olarn ``\href{https://pandas.pydata.org/}{pandas}'' kütüphanesine JavaScript için karşılığı kabul edilebilecek \href{https://danfo.jsdata.org/}{Danfo.js}'i fark ettim.
Bunu (ve buna benzer kütüphaneleri) kullanmamamın sebebi, bu kütüphanelerin C modüllerine bağımlı olması.
Yani paketi oluşturan kodun bir kısmı C dilinde yazılmış, bu kısım hedef platform için derlenmiş ve paketin içine yerleştirilmiş.
Tarayıcının içinden C modülü çağıramayacağımız/işletemeyeceğimiz için saf JavaScript ile oluşturulmuş kütüphanelerden edinmem gerekiyordu.

Bu noktada \href{https://github.com/mljs/}{mljs} imdadımıza yetişiyor.
mljs, kendi içinde makine öğrenmesi nümerik analiz araçlarını barındıran bir organizasyon ve içinde barındırdığı tüm paketler hem Node.js için hem de web tarayıcıları tarafından işletilebiliyor.
mljs'i tercih etmemin asıl sebebi bu.

mljs'in altında bulunan paketlerin tamamına \href{https://github.com/mljs/ml#readme}{GitHub sayfalarından} ulaşılabilir.
mljs'in (ve benim bu projede kullanmadığım) bazı paketleri:

\begin{description}
 \item[\href{https://github.com/mljs/kmeans}{ml-kmeans}] K-means kümeleme paketi.
 \item[\href{https://github.com/mljs/naive-bayes}{ml-naivebayes}] Naive Bayes paketi.
 \item[\href{https://github.com/mljs/cross-validation}{ml-cross-validation}] Çapraz doğrulama araçları.
 \item[\href{https://github.com/mljs/decision-tree-cart}{ml-cart}] Sınıflandırma ve regresyon ağaçları paketi.
 \item[\href{https://github.com/mljs/regression-simple-linear}{ml-regression-simple-linear}] Basit lineer regresyon paketi.
 \item[\href{https://github.com/mljs/regression-polynomial}{ml-regression-polynomial}] Polinomial regresyon paketi.
 \item[\href{https://github.com/mljs/distance}{ml-distance}] Uzaklık ve benzerlik fonksiyonları.
\end{description}

\newpage


\section{Projenin Yapısı}

Projedeki dosya ve dizinlerin hangi amaca hizmet ettiklerini şöyle açıklayayım:
\begin{verbatim}
ml-phone-price
|-- docs
|   `-- report.tex          (şu an okuduğunuz dokümanın latex kodu)
|-- model                   (makine öğrenmesi modelinin bulunduğu dizin)
|   |-- create-examples.js  (examples.csv'den rastgele örnek veri seçen betik)
|   |-- create-model.js     (modeli eğiten betik)
|   |-- helpers.js          (iki betik arasında paylaşılan bazı fonksiyonlar)
|   |-- mobile-phone-prices.csv
|   `-- mobile-phone-prices-examples.csv
|-- package.json            (bu projeyi tanımlayan bilgiler, bağımlılıklar)
|-- package-lock.json
|-- public                  (web uygulaması için gerekli statik dosyalar)
|   |-- index.html
|   |-- manifest.json
|   `-- robots.txt
|-- README.md               (beni oku dosyası)
`-- src                     (uygulamanın kaynak kodunun bulunduğu dizin)
    |-- App.js              (uygulamanın asıl komponenti)
    |-- App.test.js
    |-- examples.json -> ../model/examples.json
    |-- index.css           (temel CSS)
    |-- index.js            (react uygulamasının başlangıcı, App.js import ediliyor)
    |-- model.json -> ../model/model.json
    |-- params.json         (modelin aldığı parametrelerin listesi)
    |-- reportWebVitals.js
    `-- setupTests.js
\end{verbatim}

Bu projenin en önemli kısmı \texttt{src/App.js} olacaktır, çünkü web uygulamasının tüm kodu orada bulunuyor denebilir.
Kullanıcının parametreleri girdiği ekranın oluşturulması, makine öğrenmesi modelinin yüklenip parametrelerin modele verilip sonucun alınması, bu sonucun tekrar kullanıcıya gösterilmesi işlemleri burada yapılıyor.

``model'' dizini altındaki betiklerin doğrudan çalıştırılmasına gerek yok, projeye npm komutları şeklinde dahil edildiler.
Örneğin modeli eğitmek için \texttt{npm run trainmodel} komutu verilmesi yeterli.
Aynı şekilde test verisinden örnek seçmek için \texttt{npm run createexamples} komutu verilebilir.

\newpage


\section{Makine Öğrenmesi Modeli}

Model için kullandığım veriyi\footnote{\url{https://www.kaggle.com/iabhishekofficial/mobile-price-classification}} Kaggle'dan aldım.
Bu veride, pazardaki akıllı telefonlarının özellikleri ve fiyat sınıfları yer alıyor.
Akıllı telefonların 20 özelliği (6 boolean 14 sayı) ve bu özelliklere karşılık gelen 4 kategorili bir değer bulunuyor.
Bu kategoriler; 0 (düşük fiyat), 1 (ortalama fiyat), 2 (yüksek fiyat) ve 3 (çok yüksek fiyat) şeklinde.

Veri setinde 2000 tane örnek bulunmakta. Bunların \%25'ini test verisi olacak biçimde ayırdım. Makine öğrenmesi modeli için KNN (K-En Yakın Komşular) algoritmasını kullandım. Modelin başarımı ile ilgili bazı istatistikler:

\begin{center}
Doğruluk oranı: \%94.80
\end{center}

\begin{table}[h]
 \center
 \renewcommand{\arraystretch}{1.5}
 \begin{tabular}{ cccc|c }
 128 & 0   & 0   & 3   & \%97.71 \\
 0   & 121 & 9   & 0   & \%93.08 \\
 0   & 7   & 110 & 2   & \%92.44 \\
 4   & 0   & 1   & 115 & \%95.83 \\
 \hline
 \%96.97 & \%94.53 & \%91.67 & \%95.83 & \%94.80
 \end{tabular}
 \renewcommand{\arraystretch}{1}
 \caption{Modelin karışıklık matrisi.}
\end{table}

\begin{table}[h]
 \center
 \renewcommand{\arraystretch}{1.2}
 \begin{tabular}{ c|cc }
 & F1 & Kappa \\
 \hline
 0 & 0.973 & 0.964 \\
 1 & 0.958 & 0.945 \\
 2 & 0.921 & 0.896 \\
 3 & 0.938 & 0.916 \\
 \end{tabular}
 \renewcommand{\arraystretch}{1}
 \caption{Modelin F1 ve Kappa skorları.}
\end{table}


\section{Web Uygulaması}

Bu kısım oldukça basit, \texttt{npm run build} komutu verilerek React uygulaması dağıtılabilir hale gelir.
Bu komut proje dizinine ``build'' isimli bir dizin oluşturur.
Herhangi bir web sunucu yazılımıyla bu dizin sunulduğunda uygulamamız erişilebilir olur.

GitHub'ın bir başka özelliği olan GitHub Pages\footnote{\url{https://pages.github.com/}} ile uygulamayı yayına alabiliriz.
GitHub Pages, geliştirilen projeler veya başka amaçlar için ücretsiz statik web sayfaları sunmanıza izin veren bir servis.
Bizim projemiz için yayınlanması gereken tek şey ``build'' dizini.
Bu projede yayınlamayı yapabilmek için \texttt{npm run deploy} komutunu işletmemiz yeterli.

Uygulamaya \url{https://thesseyren.github.io/ml-phone-price/} adresinden erişilebilir.


\end{document}
